\chapter{ INTRODUCTORY CHAPTER}


\section{  Introduction}

\noindent This report gives a summary of the attaches industrial attachment from August 2018 to April 2018 at Small Scale Mining Supplies (SSMS). This report seeks to give a full outline of all the aspects learnt during the attachment period based on the various departments. Departments include Projects and Engineering Department, Manufacturing Department, Maintenance Department and Mining \& Technical Services Department.\par

\section{ Company background}

\noindent SSMS was founded in 2016 with the goal of empowering small-scale miners who were neglected in terms of efficient mining processes yet they contributed 58\% of the nation's gold production. \par

\noindent SSMS entered the market to ensure that they didn't just supply equipment but provide the best returns for the miner within 3 years. The company is small with a labour compliment of 40 people with 20 being permanent and the rest being casual labour. SSMS is a trading name for ListerFill Investments. The CEO is Collins Kendall and the MD is Laurie Cleminson. \par

\noindent Since the beginning of its operations in 2017. Small Scale Mining Supplies has supplied 12 Mines across Zimbabwe and constantly monitors and supports theses mines. It works closely with Zimbabwe Mines Department, School of Mines, Fidelity Printers.\par


\section{ Company location and layout}


\subsection{ Company Location}

\noindent SSMS is on 6 Cowden Road, Steeldale, Bulawayo. Figure 1,1 below shows the location of SSMS\par

\noindent \includegraphics*[width=6.50in, height=4.30in, keepaspectratio=false]{image42}

\noindent Figure 1.1 SSMS Location

\noindent \includegraphics*[width=4.60in, height=3.24in, keepaspectratio=false]{image43}

\noindent Figure 1.2 Company Layout


\section{ Company mission and vision}


\subsection{ Vision}

\noindent To ensure maximum, efficient and sustainable exploitation of Zimbabwe's mineral resources by small scale miners by making equipment, plants and technical services accessible to small scale miners\par

\noindent The vision shows the company goal to make sure small-scale miners have an advantage in getting what they need to be most profitable and benefit the community around them. \par


\subsection{ Mission}

\noindent Providing a total and complete solution for small scale miners while offering \par

\begin{enumerate}
\item  Sustainable employment opportunities

\item  Active skills development

\item  Empowerment of small mining communities across Zimbabwe

\item  Substantial production increase

\item  Economic participation and growth in mining communities

\item  Address current health risk

\item  Promote environmentally friendly production practises
\end{enumerate}


\subsection{ Core values}

\begin{enumerate}
\item \textbf{ }Integrity

\item  Fairness

\item  Accountability

\item  Professionalism

\item  Transparency

\item  Reliability

\item  Diligence

\item  Commitment and Teamwork
\end{enumerate}

\noindent These core values form the company philosophy under which all departments and customers must operate under. 


\section{ Organisational structure}

\noindent The organisational structure is shown as below. The Organogram show the given flow of information and command to allow for protocol. However, since the company is small, certain protocol flows are naturally disregarded for best efficiency. \par

\noindent \includegraphics*[width=6.51in, height=5.01in, keepaspectratio=false]{image44}Figure 1.3 shows the organisational structure of SSMS. 

\noindent Figure 1.3 Organogram


\subsection{ Overview of departments}


\subsubsection{ Projects and Engineering Department}

\noindent This department is responsible with carry out the design and installation projects of the plants. Its main roles include\par

\begin{enumerate}
\item  Translating customers request technically to ensure the plants meat the requirements of the client

\item  Costing of the plant including installation and commissioning of the plant. (Costing is however a multi-department exercise and includes the manufacturing department, Stores Department and Accounts Department).

\item  Dealing with transport and logistics of equipment, machinery and company personal

\item  Design and testing of new or existing concepts. 

\item  CAD/CAM rendition of all plants

\item  Responsible with the management and maintenance of all company vehicles
\end{enumerate}


\subsubsection{  Manufacturing Department}

\noindent Responsible for the manufacturing of locally manufactured products such as our head frames, windlasses, conveyors, ore bins and so forth. They are also responsible for the modification of purchased equipment as per different mine requirements.


\paragraph{ Stores Department}

\noindent Responsible for the ordering and procurement of raw materials, equipment and all company purchases. They are also responsible of job card management and inventory management. 


\paragraph{ Maintenance Department}

\noindent Responsible for the maintenance of all plant equipment and other technical challenges any mine might face. 


\paragraph{ Human Resource Management}

\noindent Responsible for all the legal work of the company including employee contracts, customer contracts and various agreements. They also deal with all labour issue and work with the accounts department to run the payroll system. 


\paragraph{ Accounts Department}

\noindent Responsible for funding of all company activities. Are also responsible for maintaining the budget and running all finance services. 


\paragraph{ Mining and Geological Services Department}

\noindent Responsible for customer prospecting and offering technical services such as sampling, gold testing and so forth. The geologists are responsible for studying ore reserves underground and hence give guidance to the mining department on the direction to be taken in mining. The survey department is responsible for pegging underground and labelling areas for easy movement underground. They also verify the supports used underground i.e. if they are strong enough or not.


\section{ Products and markets}


\subsection{ Products}

\begin{enumerate}
\item \textbf{ }Ball Mill Plants

\item  Hammer Mill Plants

\item  Wet Pan Plants

\item  Carbon in Pulp/Carbon in Leach Plant

\item  Vat Leach Plant

\item  Elution Plants

\item  Mining and Geological Services

\item  Maintenance of running plants
\end{enumerate}


\subsection{ Markets}

\noindent SSMS has a market presence across Zimbabwe mainly in Filabusi, Kadoma and Mutoko


\section{ Training programme}

\noindent The training program was made available by the Mining and Geological Services Department 


\subsection{ Training Objectives}

\noindent At the end of the attachment period the attach\'{e} must be able to 

\begin{enumerate}
\item  Understand the workings of various mining and processing operations as they apply to small scale miners

\item  Understand engineering designing of gold recovery plants. Proficiency with equipment selection and manufacturing processes. 

\item  Understand engineering maintenance systems

\item  Be familiar with the supply chain system of the company and the procurement procedures

\item  Understand technical marketing and customer relations

\item  Understand the business and finance side of the business

\item  Apply Quality Management Structures and build systems for efficient operations
\end{enumerate}

\noindent The following table 1.1 below details the training program

\noindent Table 1.1 Training Program

\begin{tabular}{|p{0.5in}|p{0.7in}|p{1.8in}|p{0.8in}|p{0.7in}|} \hline 
 & Section & Areas to be Covered & Departments & Supervisor \\ \hline 
1 & Gold Processing and Production &  Gold Mining\newline The attach\'{e} must be familiar with the process flow of gold extraction. Mainly looking at the equipment used in the extracting of gold ore from the ground, up a shaft and onto the surface. Understanding the working, design and challenges of windlasses, head frames. Understand factors to consider when designing these. Understand the working of compressors\newline  Gold Milling and Extraction of free gold\newline Understand the processing of gold from course ore to fine ore till the extraction of free gold from the ore and the disposal of the slurry. Know all the equipment and their working principles, limitations and reasons for use over others. These include but are not limited to\newline  Crushers\newline  Ball Mills\newline  Hammer Mills\newline  Concentrators\newline  Conveyors\newline  Ore Bins\newline  Gemini table\newline  Amalgam Barrel\newline  Gold Recovery\newline Understand the difference between free gold, leached gold and refectory gold. \newline Know how to obtain leached gold and understand the plant and the equipment for \newline CIP/CIL Plants\newline Vat Leach Plants\newline Understand the factors to consider when building and running leach plants.\newline Understand the following gold processes\newline  Flotation\newline  Cyanidation\newline  Elution\newline  Electrowinning\newline Understand the various onsite, chemical and laboratory test carried out at a mine. & Mining \newline Projects \& Engineering &  \\ \hline 
2 & Business, Accounts and Stores & Understand the business model identifying its challenges and benefits. \newline Know all valuable clients and partners\newline Be able to Cost all company activities especially those dealing plant installation\newline Be able to be a project manager and handle all the logistics involved and devising appropriate systems for efficient resource allocation. \newline Understand the operations of the stores department including procurement, issuing and receiving goods, and inventory management.\newline Understand how Pastel works \newline Understand technical marketing & Accounts, Projects, HR &  \\ \hline 
3 & Engineering Design & Use CAD/CAM software to model all plants and provide engineering drawings. \newline Employ Quality Assurance measures to ensure optimum quality of products. Consequently, help employ and design quality management systems in all departments.\newline Design and run a data storage system   & Projects &  \\ \hline 
4 & Maintenance Management & Be able to set up and run a maintenance schedule\newline Understand the principles of problem diagnostics and being able to find a solution\newline Understand the underlying working principle of every piece of equipment & Maintenance &  \\ \hline 
\end{tabular}


\section{ Summary }

\noindent The chapter gave an introduction to SSMS. It mentioned in detail the company profile which includes the background, location and layout, vision and mission statements, organisational structure, products and markets. It also gave the outline of the training program for the attachment period. This chapter gave the reader an overview of all activities and operations of SSMS. 
