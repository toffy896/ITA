\chapter{ GOLD PROCESSING}


\section{  Introduction}

\noindent This chapter details the processing of ore to pure gold in the various plants. This chapter details the different stages that are available in the gold processing flow and the various equipment used per stage depending on customer requirements. This chapter will however focus on the following equipment which are often used by small scale miners and which SSMS often sells. \par


\section{ Processing Plant Organisational Structure}

\noindent Small Scale Mining Supplies does not determine the organisational structure of any of the mines it installs. However, this is the organisational structure that SSMS recommends and is most used by small scale miners is in Figure 2.1\par

\noindent \includegraphics*[width=4.93in, height=3.53in, keepaspectratio=false]{image45}

\noindent Figure 2.1 Plant Proposed Structure

\noindent Other services and personnel are outsourced as per demand. These include instrumentation technicians.


\section{ Plant Layouts}


\subsection{ Standard Plant Layout }

\noindent The following plant layout shows the arrangements of the milling plants SSMS provide. The one below (Figure 2.2) is for a 1-ton ball mill. However regardless of the type of crusher, mill and so forth the layout is the same. Differences occur when the plant offered is a modification of an existing plant or there is reason to change the layout such as space constraints or already existing infrastructure. \par

\noindent \includegraphics*[width=6.52in, height=4.42in, keepaspectratio=false]{image46}

\noindent Figure 2.2 Plant Layout


\section{ Process Plant Flow Chart}

\noindent \textbf{\includegraphics*[width=6.01in, height=2.51in, keepaspectratio=false]{image47}}

\noindent Figure 2.3 Plant Process Flow Chart


\subsection{ Crusher Section}

\noindent SSMS supplies Jaw Crushers as they are easier for small scale miners to use and they have various spares in the country. \par

\noindent This is the initial stage in ore processing. The jaw crusher is a single toggle machine. The jaw crusher reduces ore to 30mm. It is attached to a feed chute that also works to filter small stones and sand of size $<<$125mm to avoid chocking the crusher. Most mines only need a primary crusher. However, depending on the type of ore and blasting method, a secondary crusher may be required to make the crushed stone not over work the crusher or choke the mills. \par


\subsubsection{ Working Principle of the Crusher}

\noindent Jaw crushers are equipped with one fixed and one moveable crushing jaw (moving jaw), both of which support crushing plates (wearing parts) in several versions. They form a wedge-shaped crushing zone as shown Figure 2.4. The walls of the crushing zone are made of replaceable wearing sheets. Crushing is done between the two crushing jaws. The moving jaw of the single toggle jaw crusher moves elliptically. The crushing force is produced by an eccentric shaft. Then it is transferred to the crushing zone via a toggle plate system and supported by the back wall of the housing of the machine. Spring pulling rods keep the whole system in a condition of nonpositive connection. Centrifugal masses on the eccentric shaft serve as compensation for heavy loads. A flywheel is provided in the form of a pulley. Due to the favourable angle of dip between the crushing jaws, the feeding material can be reduced directly after entering the machine. The final grain size distribution is influenced by both, the adjustable crusher setting and the suitability of the tooth form selected for the crushing plates. All parts of the adjusting system are accessible, maintenance-free and easy to replace. All of the toggle plates are provided with replaceable tempered pressure bars on both sides, which work according to the move on rolling contact principle. They are maintenance-free. A lubrication system ensures sufficient grease supply to all other bearing points. \par

\noindent \textbf{\includegraphics*[width=6.46in, height=4.33in, keepaspectratio=false]{image48}}

\noindent Figure 2.4 Crusher Mechanism

\noindent The crushing plates are made of highly wear-resistant austenitic manganese steel casting and are constructed in such a way that they can easily be turned after having loosened the clamping parts. There are easily-replaceable side wedges are made of wear-resistant sheet steel or steel casting.\par

\noindent \includegraphics*[width=6.52in, height=4.98in, keepaspectratio=false]{image49}\textbf{}

\noindent Figure 2.5 Crusher Wearables


\subsubsection{ Jaw Crusher Specification}

\noindent Type:  Single Toggle 8 inches by 5 inches

\noindent Feeding Size:120mm

\noindent Size of final grain:30mm

\noindent Reduction Ratio: 1:4

\noindent Required Power:5Kw

\noindent Motor RPM:1476rpm


\subsection{ Conveyor Belts}

\noindent A conveyor belt is a mechanical apparatus consisting of a continuous moving belt that transports materials or packages from one place to another. In the plant, they are used to transport ore from primary crushing to tertiary crushing. Only two types of conveyor belts are used in the plant, 3ply and 4ply belts. 4ply belts are used in primary crushing because they carry heavier ore (+150mm) hence they have a greater impact on the belt. 3ply belts are used in the secondary and tertiary section because they are lighter.\par


\subsubsection{  Belt specifications}

\noindent The conveyor belts in the plant range from 300mm to 600mm in width. They are made of layers of fibre impregnated with rubber composition and have a thin layer of rubber on the faces. They are highly suitable when carrying damp material. Average allowable stress is 1.75Mpa\par


\subsection{ Milling Section}

\noindent SSMS will provide either a Hammer Mill or a Ball Mill The milling section reduces ore size from 3mm to 45µm.\par


\subsubsection{ Ball Mill}


\subparagraph{ Operating Procedure of Ball Milling.}

\noindent The Ball Mill works by using abrasion and impact to crush the charge and ground it to smaller size. A Ball Mill is a cylindrical drum which has abrasion resistant lining ( though wearable). The ball mill is attached to a rotating shaft or motor system. AS it rotates the mill balls inside it hit against the charge thus grinding it\par

\noindent \includegraphics*[width=4.46in, height=3.16in, keepaspectratio=false]{image50}

\noindent Figure 2.6 Motion of charge in a mill 


\subparagraph{ Calculations for a ball mill}


{\bf  Critical Speed}

\noindent A ball mill critical speed is the speed at which the centrifugal forces equal gravitational forces at the mill shells inside surface and no balls will fall from its position onto the shell. It is given by
\[N_c=76.6(D^{-0.5})\] 
Where D is the mill effective inside diameter in feet

\noindent And $N_c$ is the critical speed in rpm

\noindent The mill speed is consequently described as a fraction of the critical speed. Different mill speeds affect the milling process and determine efficiency of the mill. The following diagram illustrates this. The black dot shows the centre of gravity for charge.\par

\noindent \includegraphics*[width=6.47in, height=1.79in, keepaspectratio=false]{image51}

\noindent Figure 2.7 Effect of Mill Speed on Milling

\noindent Ball Mills often work at a recommended speed of between 65-79\% of the critical speed

\noindent A ball mill has an efficiency of 80-89\%


{\bf  Ball Load}

\noindent It is very important to have the correct amount of ball load within the ball mill to ensure thorough and efficient milling. The formula is given below:
\[\%loading=113-126\frac{H}{D}\] 
Where H- distance from top of mill to top of charge

\noindent D -- External diameter of mill


\subparagraph{ Ball Mill Specification for SSMS}

\noindent Type: 1 Ton per Hour

\noindent Feeding Size:30mm

\noindent Size of final grain:150 micro mm

\noindent Required Power:5kW

\noindent Ball Load:300kg


\subsubsection{ Hammer Mill}


\subparagraph{ Operating procedure}

\noindent A hammer mill is essentially a steel drum containing a horizontal rotating shaft or drum on which hammers are mounted. The hammers are free to swing on the ends of the cross, or fixed to the central rotor. \par

\noindent \includegraphics*[width=4.79in, height=2.68in, keepaspectratio=false]{image52}

\noindent Figure 2.8 Hammer Mill Hammers on Shaft

\noindent The rotor is spun at a high speed inside the drum while material is fed into a feed hopper. The material is impacted by the hammer bars and is thereby shredded and expelled through screens in the drum of a selected size.\par

\noindent \includegraphics*[width=6.41in, height=5.36in, keepaspectratio=false]{image53}

\noindent Figure 2.9 Hammer Mill Feed Action

\noindent There are various grates or screens used depending on the output size required. Each mine is different in terms of the ore that it produces

\noindent \includegraphics*[width=5.57in, height=4.31in, keepaspectratio=false]{image54}

\noindent Figure 2.10 Grate Mesh.

\noindent 

\noindent \eject \textbf{\textit{}}