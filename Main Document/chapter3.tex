\chapter{ GOLD ORE EXTRACTION}


\section{  Introduction}

\noindent The following chapter outlines the processes and equipment that is provided by SSMS to assist small scale miners in extracting the ore from the ground to the surface. These include hoists, compressors, jack hammers and windlass \par


\section{ Windlass}

\noindent Windlass is used to lift ore from the shaft. They are manually operated but only work upto a depth of 30m. Anything more becomes dangerous. Most miners who are starting off prefer this as it is cheap hoisting option. The bucket carries about 100kg of ore.  Figure 3.1 shows the windlass \par

\noindent \includegraphics*[width=4.00in, height=3.81in, keepaspectratio=false, trim=0.00in 0.00in 1.61in 0.00in]{image55}

\noindent Figure 3.1 Windlass


\section{ Hoist}

\noindent All head frames make use of a semi automatic hoisting system. Most hoists supplied are for 1 Ton weights, moving at a speed of 11m per minute with power requirements of 5kw. The winch/hoist is responsible for the pulling of the rope and ore buckets from both the vertical and the inclined shafts. The following make up the winch system\par

\begin{enumerate}
\item \begin{enumerate}
\item \begin{enumerate}
\item \begin{enumerate}
\item  Electric motor -- it is connected to electricity and provides the drive for the whole system. It is connected to electrolyte deepers as well. Motor speed at takeoff is low giving the hoist high torque so as to overcome any resistance to motion.

\item  Gearbox -- it reduces the rotational speed from the motor before it reaches the hoist drum. This is because the rope is not supposed to move at high speeds for safety of the ore or humans being transported.

\item  Drum -- this is where the rope is wound. If it is a single drum, ropes have to be overlay. The rope is clamped inside the drum by an average of 3 u-clamps.

\item  Control center -- this is where the hoist driver sits. There is a directional control lever, emergency stop button, bell interface, brakes, dead man`s switch and various buttons which indicate a fault when it occurs.

\item  Electrolyte deepers - The main purpose of the deepers is to enable the skip to move at high torque and less speed at takeoff.
\end{enumerate}
\end{enumerate}
\end{enumerate}

\item  Emergency stop switch - This is a button on the driver `s control panel. It is used when a fault occurs or in the case of an emergency. When pressed, the hoist system shuts down.

\item  Dead man`s switch - As the driver is operating the hoist, he is required to continuously step on the dead man`s switch. In case of a sudden predicament, he may lose control of the lever and the fastest way to stop the hoist will be to remove his foot on the switch. This immediately stops the hoist system.
\end{enumerate}


\subsection{ Ropes for the hoist }

\noindent Ropes are an integral part of the hoist system. Wire ropes are used for hoisting operations over fibre ropes because\par

\begin{enumerate}
\item  they are stronger,

\item  last longer 

\item  resistant to abrasion. 
\end{enumerate}

\noindent Before a rope is chosen, construction, characteristics and specifications should be known.


\subsubsection{ Rope construction}

\noindent The basic component of a wire rope is a wire. The wire may be made of steel, iron or other metal in various sizes. Wire ropes are designated by the number of strands in a rope and the number of wires in a strand. Figure 3.6 below shows the fabrication of a wire rope.\par

\noindent \includegraphics*[width=3.26in, height=2.87in, keepaspectratio=false]{image56}

\noindent Figure 3.6: Fabrication of a wire rope 

\noindent 

\noindent The wire core supports the strands laid around it. The three types of cores are wire strand, fiber and independent wire rope.\par

\begin{enumerate}
\item  A fiber core is flexible, cushions sudden strains and acts as an oil reservoir for lubrication of rope. 

\item  A wire strand is more heat resistant compared to a fiber core, adds 15\% more strength to the rope though it offers less flexibility.

\item  An independent wire rope is a separate wire rope over which the main strands of a rope are laid. It strengthens the rope, offers more resistance to heat and supports the rope against crushing.
\end{enumerate}

\noindent Plow steel wire rope is used for hoisting, hauling and logging. It is unusually tough and strong. The tensile strength ranges from 220 000-240 000 psi.\par


\subsubsection{ Wire rope selection}

\noindent A lot of factors have to be considered before choosing a rope. One rope cannot satisfy all requirements hence some qualities have to be compromised and counteracted during maintenance. The basic requirements are listed below.\par

\begin{enumerate}
\item  Tensile strength -- strength necessary to withstand a certain maximum load applied to a rope

\item  Crushing strength -- strength necessary to withstand compressive and squeezing forces that distort cross section of a rope as it runs over sheaves, rollers and hoist drums when under heavy load. Regular lang lay distorts less in these situations than lang lay.

\item  Fatigue resistance -- ability to withstand the constant bending and flexing of wire rope that run continuously on sheave ropes and hoist drums. It is important when rope is run at high speeds. Lang lay ropes are best for this application.

\item  Abrasion resistance -- the ability to withstand the gradual wearing away of outer layer on ropes. It is due to constantly carrying heavy loads at high speeds. Ropes with larger diameter of wires in outer strands can withstand abrasion better. It also depends on type of metal , improved plow steel is the best choice.

\item  Corrosion resistance -- ability to resist dissolution of rope due to chemical attack by moisture in the atmosphere or working environment. This can be counteracted by painting, galvanising or lubrication.
\end{enumerate}


\subsubsection{ Main service ropes}

\noindent As stated above, rope selection differs depending on application. Table 3.1 below shows the comparison of the main service ropes.\par

\noindent Table 3.1 Comparison of main service ropes

\begin{tabular}{|p{1.3in}|p{0.8in}|p{0.8in}|p{1.1in}|} \hline 
 & 4 m Gantry & 7.5m Gantry & Incline Shaft \\ \hline 
Diameter of rope & 20mm & 20mm & 26mm \\ \hline 
Breaking force & 264.89kN & 254.87kN & 361.65kN \\ \hline 
Type of lubrication & Noxal super 8 & Noxal super 8 & Noxal super 8 \\ \hline 
Tensile strength of steel & 1800Mpa & 1800Mpa & 1960Mpa \\ \hline 
Length of rope & 120m & 120m & 120m \\ \hline 
Mass per length & 2.44kg/m & 2.44kg/m & 2.70kg/m \\ \hline 
Lay length & 170mm & 135mm & 135mm \\ \hline 
Type of lay & RHL & RHL & RHL \\ \hline 
\end{tabular}


\subsection{ Rope replacement }

\noindent Ropes are expected to last 2years by the SI 109. If by any reason they are found to be suitable for use after 2 years, the government inspector gives the mine an exemption which allows the mine to continue using the rope.\par


\section{ Vertical Head Frame}

\noindent SSMS makes vertical headframes of the height dimensions 4m and 7.5m. The 4m Vertical head frame is capable of depths of up to 90m while the 7.5m headgear is capable of upto 120m , both depending on the rope length. The 4 m headgear is capable of upto 1 ton including the bucket weight and rope weight. Therefore, the effective ore that is hoisted up is round about 250kg per lift. Hoisting ore from 60m deep rough takes about 8minutes to lower , fill, raise and offload the bucket.\par 

\subsection{The parts of a head gear}


\subsubsection{ Shaft Doors and Bottom Platform}

\noindent These work more as a safety feature. The shaft doors are closed at all times except when men or ore is coming into or out of the shaft. This prevents injuries and fatalities caused by people falling into the shaft or by objects falling onto people in the shaft. The shaft doors are made strong enough for a loaded bucket and two men to stand on top of. This is done so that the bucket has a resting place before being offloaded and it prevents the bucket from falling back into the shaft. The platform allows for movement around the gantry. \par

\subsubsection{ Sheave Wheel}

\noindent The sheave wheel acts to direct the rope along a certain path and works to change the rope direction. The bearings on a sheave wheel need to regularly checked to avoid wear and tear of the shaft and possible jams during lifting operations. \par


\subsubsection{ Top Platform with guiding rails}

\noindent The top platform works for maintenance of the sheave wheel and greasing of the sheave wheel. It is made with rails at the top to prevent accidentally falling off from the platform. \par


\section{ Inclined Head Frame}

\noindent The inclined head frame is made depending on the angle of the shaft. Inclined shafts have the same components as a vertical head frame with some addition parts.


\subsection{ Skipper Train on rails}

\noindent The Skipper train carries the ore from underground and onto the surface. They have a tipping mechanism to allow for easy tipping\par

\noindent \includegraphics*[width=4.50in, height=3.37in, keepaspectratio=false]{image57}

\noindent Figure 3. Skipper Train for inclined shaft


\subsection{ Hopper}

\noindent 7 Ton Hopper is for storage of ore. The cocopan tips over into the hopper. This helps also feel the tractors which transport the ore from the mining shafts to the processing plant.  \par


\section{ Compressor}

\noindent Compressors issued to mines under the SSMS are within the range of 145cfm to 250cfm and pressure 9 bars to 18 bars. They are used to power the jack hammer. Doosan is the choosen brand due to warranty. \par

\noindent \includegraphics*[width=5.06in, height=3.39in, keepaspectratio=false, trim=0.44in 0.51in 1.04in 0.59in]{image58}

\noindent Figure Air Compressor


\subsection{ Compressor maintenance}

\noindent \textbf{}

\noindent Compressor maintenance is done depending on hourly service. The hourly service maintenance are 2000hrs, 4000hrs, 6000hrs, 8000hrs and the cycle begins again.\par

\begin{enumerate}
\item  2000hr service -- inspect and replace air and oil filters

\item  4000hr service -- dismantling and cleaning of all valves. Replacing all damaged and worn out parts. Fitting new O-rings on the unloading plungers.

\item  6000hr service -- repeat 2000hr service

\item  8000hr service -- dismantle, clean and inspect all valves. Replace all valve springs, valve discs and buffer discs, even if the faultless. Fit new O-rings on the unloading plungers. Dismantle and clean the regulating valves, including their air filter. Drain the oil and clean the interior of the crankcase. Also clean the oil strainer. Refill the oil. Clean the lubricator sump and sight-glasses.\par
\end{enumerate}


\section{ Jack hammer}

\noindent Drilling is the creation of holes in the hard rock. This is done by pneumatic machines known as jack hammers. The drilling rate of jack hammers is 5.56mm/s. Jack hammers use compressed air for turning the drill bit and moving the jack arm. Water is used for cooling jack hammers during operation since the drilling process generates a lot of heat. 40 holes are drilled per shift; this is equivalent to 40 tonnes of ore after blasting. Each hole is 2m. Lubrication is done on the drill bit so as to minimize friction between it and the casing. During operation, the jack hammers produce high vibrations due to resistance offered by the hard rock. Figure 3.below shows the schematic diagram of a jack hammer.\par

\noindent 

\noindent \includegraphics*[width=6.50in, height=2.43in, keepaspectratio=false]{image59}

\noindent Figure 3. Schematic diagram of a jack hammer.

\noindent Production Monitoring


\section{ Summary}

\noindent The chapter detailed the equipment used for gold ore extraction. This equipment is supplied to small scale miners in order to extract the ore in the most cost efficient and safest way. \par